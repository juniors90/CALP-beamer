\documentclass[xetex, 11pt,spanish]{beamer}
\usepackage[spanish]{babel}
\usetheme{Madrid}

\makeatletter
\def\th@mystyle{%
	\normalfont % body font
	\setbeamercolor{block title example}{bg=orange,fg=white}
	\setbeamercolor{block body example}{bg=orange!20,fg=black}
	\def\inserttheoremblockenv{exampleblock}
}
\makeatother
\theoremstyle{mystyle}
\newtheorem*{remark}{Remark}
\usepackage{agda}
\usepackage{fontspec}
\usepackage{unicode-math}

\setmainfont{DejaVu Serif}
\setsansfont{DejaVu Sans}
\setmainfont{XITS}
\setmathfont{XITS Math}
% Alternatively:
%\setmathfont{DejaVu Math TeX Gyre}
\setmonofont{DejaVu Sans Mono}
\usepackage{latex/agda}
\usepackage{catchfilebetweentags}
\begin{document}
\author{Ferreira Juan David}
	\title{Agda: Un lenguaje con tipos dependientes desde la práctica}
	%\subtitle{}
	%\logo{}
	%\institute{}
	%\date{}
	%\subject{}
	%\setbeamercovered{transparent}
	%\setbeamertemplate{navigation symbols}{}
	\begin{frame}[plain]
		\maketitle
	\end{frame}
  \begin{frame}
    \frametitle{Introducción}
    La motivación de este trabajo es analizar el paper 
    "Nominal Sets in Agda. A Fresh and Immature Mechanization" (Miguel Pagano y Jose E. Solsona)
    a partir del cual presentan el desarrollo de una nueva formalización 
    de un conjunto Nominal en Agda.
    
    Este trabajo contribuye a una mejor comprensión de los conjuntos \textbf{Nominales} y 
    aporta una forma de
    probar sistemas de tipos basados en lógica nominal.
\end{frame}
  \begin{frame}
    \frametitle{Semigrupo, Monoide y Grupo}
        Dado un conjunto $G$, $G\not = \emptyset$, una operación binaria
        es una función $$\ast : G \times G \to G.$$
    
        Un semigrupo es $G$ un conjunto, $G\not = \emptyset$ junto con una operación 
        binaria $\ast : G \times G \to G$ que es \textbf{asociativa}. Es decir,
        $$(a \ast b) \ast c = a \ast (b \ast c); \forall a, b, c \in G.$$
    
        Un monoide es un semigrupo $G$ un conjunto, si existe un \textbf{elemento identidad}
        (a ambos lados) $e \in G$ tal que $$\forall a \in G | a \ast e = e \ast a = a.$$
    
        Un grupo es un monoide $G$, tal que todo elemento posee \textbf{inverso} vía $\ast$.
        Es decir, $$\forall a \in G : \exists a^{-1} \in G | a \ast a^{-1} = a^{-1} \ast a = e.$$
\end{frame}
  \begin{frame}\frametitle{Semigrupo}

\begin{code}%
\>[0]\AgdaKeyword{record}\AgdaSpace{}%
\AgdaPrimitive{Semigroup}\AgdaSpace{}%
\AgdaGeneralizable{c}\AgdaSpace{}%
\AgdaGeneralizable{ℓ}\AgdaSpace{}%
\AgdaSymbol{:}\AgdaSpace{}%
\AgdaRecord{Set}\AgdaSpace{}%
\AgdaSymbol{(}\AgdaBound{suc}\AgdaSpace{}%
\AgdaSymbol{(}\AgdaSpace{}%
\AgdaGeneralizable{c}\AgdaSpace{}%
\AgdaOperator{\AgdaPrimitive{⊔}}\AgdaSpace{}%
\AgdaGeneralizable{ℓ}\AgdaSpace{}%
\AgdaSymbol{)}\AgdaSpace{}%
\AgdaSymbol{)}\AgdaSpace{}%
\AgdaKeyword{where}\<%
\\
\>[0][@{}l@{\AgdaIndent{0}}]%
%\>[4]\AgdaInductiveConstructor{zero}%
\>[4]\AgdaKeyword{field}\<%
\\
\>[4][@{}l@{\AgdaIndent{0}}]%
%
\>[8]\AgdaInductiveConstructor{Carrier}%
\>[12]\AgdaSpace{}\AgdaSpace{}\AgdaSpace{}
\AgdaSpace{}\AgdaSpace{}\AgdaSpace{}\AgdaSpace{}%
\AgdaSpace{}\AgdaSpace{}\AgdaSymbol{:}\AgdaSpace{}%
\AgdaRecord{Set}\AgdaSpace{}
\AgdaGeneralizable{c}\AgdaSpace{}%\<%
\\
\>[4][@{}l@{\AgdaIndent{0}}]%
%
\>[8]\AgdaOperator{\AgdaFunction{\AgdaUnderscore{}≈\AgdaUnderscore{}}}\AgdaSpace{}%
\>[14]\AgdaSymbol{:}\AgdaSpace{}%
\AgdaFunction{Rel}\AgdaSpace{}%
\AgdaField{Carrier}\AgdaSpace{}%
\AgdaGeneralizable{ℓ}\AgdaSpace{}%
\<%
\\
\>[4][@{}l@{\AgdaIndent{0}}]%
%
\>[10]\AgdaOperator{\AgdaFunction{\AgdaUnderscore{}∙\AgdaUnderscore{}}}\AgdaSpace{}%
\>[14]\AgdaSymbol{:}\AgdaSpace{}%
\AgdaOperator{\AgdaFunction{Op₂}}\AgdaSpace{}\AgdaSpace{}%
\AgdaField{Carrier}\AgdaSpace{}%
\<%
\\
\>[4][@{}l@{\AgdaIndent{0}}]%
%
\>[8]\AgdaInductiveConstructor{isSemigroup}%
\>[14]\AgdaSymbol{:}\AgdaSpace{}%
\AgdaFunction{IsSemigroup}\AgdaSpace{}%
\AgdaOperator{\AgdaFunction{\AgdaUnderscore{}≈\AgdaUnderscore{}}}\AgdaSpace{}%
\AgdaOperator{\AgdaFunction{\AgdaUnderscore{}∙\AgdaUnderscore{}}}\AgdaSpace{}%
\end{code}

\end{frame}

  \begin{frame}\frametitle{Monoide}

\begin{code}%
\>[0]\AgdaKeyword{record}\AgdaSpace{}%
\AgdaPrimitive{Monoid}\AgdaSpace{}%
\AgdaGeneralizable{c}\AgdaSpace{}%
\AgdaGeneralizable{ℓ}\AgdaSpace{}%
\AgdaSymbol{:}\AgdaSpace{}%
\AgdaRecord{Set}\AgdaSpace{}%
\AgdaSymbol{(}\AgdaBound{suc}\AgdaSpace{}%
\AgdaSymbol{(}\AgdaSpace{}%
\AgdaGeneralizable{c}\AgdaSpace{}%
\AgdaOperator{\AgdaPrimitive{⊔}}\AgdaSpace{}%
\AgdaGeneralizable{ℓ}\AgdaSpace{}%
\AgdaSymbol{)}\AgdaSpace{}%
\AgdaSymbol{)}\AgdaSpace{}%
\AgdaKeyword{where}\<%
\\
\>[0][@{}l@{\AgdaIndent{0}}]%
%\>[4]\AgdaInductiveConstructor{zero}%
\>[4]\AgdaKeyword{field}\<%
\\
\>[4][@{}l@{\AgdaIndent{0}}]%
%
\>[8]\AgdaInductiveConstructor{Carrier}%
\>[12]\AgdaSpace{}\AgdaSpace{}\AgdaSpace{}
\AgdaSpace{}\AgdaSpace{}\AgdaSpace{}\AgdaSpace{}%
\AgdaSpace{}\AgdaSpace{}\AgdaSymbol{:}\AgdaSpace{}%
\AgdaRecord{Set}\AgdaSpace{}
\AgdaGeneralizable{c}\AgdaSpace{}%\<%
\\
\>[4][@{}l@{\AgdaIndent{0}}]%
%
\>[8]\AgdaOperator{\AgdaFunction{\AgdaUnderscore{}≈\AgdaUnderscore{}}}\AgdaSpace{}%
\>[14]\AgdaSymbol{:}\AgdaSpace{}%
\AgdaFunction{Rel}\AgdaSpace{}%
\AgdaField{Carrier}\AgdaSpace{}%
\AgdaGeneralizable{ℓ}\AgdaSpace{}%
\<%
\\
\>[0][@{}l@{\AgdaIndent{0}}]%
%
\>[10]\AgdaOperator{\AgdaFunction{\AgdaUnderscore{}∙\AgdaUnderscore{}}}\AgdaSpace{}%
\>[14]\AgdaSymbol{:}\AgdaSpace{}%
\AgdaOperator{\AgdaFunction{Op₂}}\AgdaSpace{}\AgdaSpace{}%
\AgdaField{Carrier}\AgdaSpace{}%
\<%
\\
\>[4][@{}l@{\AgdaIndent{0}}]%
%
\>[10]\AgdaOperator{\AgdaFunction{ε}}\AgdaSpace{}%
\>[14]\AgdaSymbol{:}\AgdaSpace{}%
\AgdaField{Carrier}\AgdaSpace{}%
\<%
\\
\>[4][@{}l@{\AgdaIndent{0}}]%
%
\>[8]\AgdaInductiveConstructor{isMonoid}%
\>[14]\AgdaSymbol{:}\AgdaSpace{}%
\AgdaFunction{IsMonoid}\AgdaSpace{}%
\AgdaOperator{\AgdaFunction{\AgdaUnderscore{}≈\AgdaUnderscore{}}}\AgdaSpace{}%
\AgdaOperator{\AgdaFunction{\AgdaUnderscore{}∙\AgdaUnderscore{}}}\AgdaSpace{}%
\AgdaOperator{\AgdaFunction{ε}}\AgdaSpace{}%
\end{code}

\end{frame}

  \begin{frame}\frametitle{Grupo}

\begin{code}%
\>[0]\AgdaKeyword{record}\AgdaSpace{}%
\AgdaPrimitive{Group}\AgdaSpace{}%
\AgdaGeneralizable{c}\AgdaSpace{}%
\AgdaGeneralizable{ℓ}\AgdaSpace{}%
\AgdaSymbol{:}\AgdaSpace{}%
\AgdaRecord{Set}\AgdaSpace{}%
\AgdaSymbol{(}\AgdaBound{suc}\AgdaSpace{}%
\AgdaSymbol{(}\AgdaSpace{}%
\AgdaGeneralizable{c}\AgdaSpace{}%
\AgdaOperator{\AgdaPrimitive{⊔}}\AgdaSpace{}%
\AgdaGeneralizable{ℓ}\AgdaSpace{}%
\AgdaSymbol{)}\AgdaSpace{}%
\AgdaSymbol{)}\AgdaSpace{}%
\AgdaKeyword{where}\<%
\\
\>[0][@{}l@{\AgdaIndent{0}}]%
%\>[4]\AgdaInductiveConstructor{zero}%
\>[4]\AgdaKeyword{field}\<%
\\
\>[4][@{}l@{\AgdaIndent{0}}]%
%
\>[8]\AgdaInductiveConstructor{Carrier}%
\>[12]\AgdaSpace{}\AgdaSpace{}\AgdaSpace{}
\AgdaSpace{}\AgdaSpace{}\AgdaSpace{}\AgdaSpace{}%
\AgdaSpace{}\AgdaSpace{}\AgdaSymbol{:}\AgdaSpace{}%
\AgdaRecord{Set}\AgdaSpace{}
\AgdaGeneralizable{c}\AgdaSpace{}%\<%
\\
\>[4][@{}l@{\AgdaIndent{0}}]%
%
\>[8]\AgdaOperator{\AgdaFunction{\AgdaUnderscore{}≈\AgdaUnderscore{}}}\AgdaSpace{}%
\>[14]\AgdaSymbol{:}\AgdaSpace{}%
\AgdaFunction{Rel}\AgdaSpace{}%
\AgdaField{Carrier}\AgdaSpace{}%
\AgdaGeneralizable{ℓ}\AgdaSpace{}%
\<%
\\
\>[0][@{}l@{\AgdaIndent{0}}]%
%
\>[10]\AgdaOperator{\AgdaFunction{\AgdaUnderscore{}∙\AgdaUnderscore{}}}\AgdaSpace{}%
\>[14]\AgdaSymbol{:}\AgdaSpace{}%
\AgdaOperator{\AgdaFunction{Op₂}}\AgdaSpace{}\AgdaSpace{}%
\AgdaField{Carrier}\AgdaSpace{}%
\<%
\\
\>[4][@{}l@{\AgdaIndent{0}}]%
%
\>[10]\AgdaOperator{\AgdaFunction{ε}}\AgdaSpace{}%
\>[14]\AgdaSymbol{:}\AgdaSpace{}%
\AgdaField{Carrier}\AgdaSpace{}%
\<%
\\
\>[0][@{}l@{\AgdaIndent{0}}]%
%
\>[10]\AgdaOperator{\AgdaFunction{\AgdaUnderscore{}⁻¹\AgdaUnderscore{}}}\AgdaSpace{}%
\>[14]\AgdaSymbol{:}\AgdaSpace{}%
\AgdaOperator{\AgdaFunction{Op₁}}\AgdaSpace{}\AgdaSpace{}%
\AgdaField{Carrier}\AgdaSpace{}%
\<%
\\
\>[4][@{}l@{\AgdaIndent{0}}]%
%
\>[8]\AgdaInductiveConstructor{isGroup}%
\>[14]\AgdaSymbol{:}\AgdaSpace{}%
\AgdaFunction{IsGroup}\AgdaSpace{}%
\AgdaOperator{\AgdaFunction{\AgdaUnderscore{}≈\AgdaUnderscore{}}}\AgdaSpace{}%
\AgdaOperator{\AgdaFunction{\AgdaUnderscore{}∙\AgdaUnderscore{}}}\AgdaSpace{}%
\AgdaOperator{\AgdaFunction{ε}}\AgdaSpace{}%
\AgdaOperator{\AgdaFunction{\AgdaUnderscore{}⁻¹\AgdaUnderscore{}}}\AgdaSpace{}%
\end{code}

\end{frame}

  %\begin{frame}\frametitle{Acci/'on}
    \begin{block}{Acción}
        Una acción de un grupo $G$ en un conjunto $S$ es una función 
        $\cdot : G \times S \to S$, $(g, s) \mapsto g \cdot s $  tal que
        \begin{itemize}
            \item[$i)$] $(gh)\cdot s = g \cdot (h \cdot s); \forall g, h \in G$ y $\forall s \in S$.
            \item[$ii)$] $e \cdot s = s; \forall s \in S$, donde $e$ es el \textbf{elemento identidad} de $G$.
        \end{itemize}
    \end{block}
\begin{AgdaAlign}
\begin{code}%
\>[0]\AgdaKeyword{record}\AgdaSpace{}%
\AgdaRecord{IsAction}\AgdaSpace{}%
\AgdaSymbol{(}\AgdaBound{F}\AgdaSpace{}%
\AgdaSymbol{:}\AgdaSpace{}%
\AgdaRecord{Func}\AgdaSpace{}%
\AgdaSymbol{(}\AgdaFunction{G.setoid}\AgdaSpace{}%
\AgdaOperator{\AgdaFunction{×ₛ}}\AgdaSpace{}%
\AgdaBound{A}\AgdaSymbol{)}\AgdaSpace{}%
\AgdaBound{A}\AgdaSymbol{)}\AgdaSpace{}%
\AgdaSymbol{:}\AgdaSpace{}%
\AgdaPrimitive{Set}\AgdaSpace{}%
\AgdaOperator{\AgdaFunction{\AgdaUnderscore{}}}\AgdaSpace{}%
\AgdaKeyword{where}\<%
\\
\>[0][@{}l@{\AgdaIndent{0}}]%
%
\>[6]\AgdaOperator{\AgdaFunction{\AgdaUnderscore{}∙ₐ\AgdaUnderscore{}}}\AgdaSpace{}%
\AgdaSymbol{:}\AgdaSpace{}%
\AgdaField{Carrier}\AgdaSpace{}%
\AgdaBound{G}\AgdaSpace{}%
\AgdaSymbol{→}\AgdaSpace{}%
\AgdaField{Carrier}\AgdaSpace{}%
\AgdaBound{A}\AgdaSpace{}%
\AgdaSymbol{→}\AgdaSpace{}%
\AgdaField{Carrier}\AgdaSpace{}%
\AgdaBound{A}\<%
\\
%
\>[6]\AgdaOperator{\AgdaFunction{\AgdaUnderscore{}∙ₐ\AgdaUnderscore{}}}\AgdaSpace{}%
\AgdaBound{g}\AgdaSpace{}%
\AgdaBound{x}\AgdaSpace{}%
\AgdaSymbol{=}\AgdaSpace{}%
\AgdaField{Func.f}\AgdaSpace{}%
\AgdaBound{F}\AgdaSpace{}%
\AgdaSymbol{(}\AgdaBound{g}\AgdaSpace{}%
\AgdaOperator{\AgdaInductiveConstructor{,}}\AgdaSpace{}%
\AgdaBound{x}\AgdaSymbol{)}\<%
\\
%
\>[6]\AgdaKeyword{field}\<%
\\
\>[6][@{}l@{\AgdaIndent{0}}]%
\>[8]\AgdaField{idₐ}\AgdaSpace{}%
\AgdaSymbol{:}\AgdaSpace{}%
\AgdaSymbol{∀}\AgdaSpace{}%
\AgdaBound{x}\AgdaSpace{}%
\AgdaSymbol{→}\AgdaSpace{}%
\AgdaFunction{ε}\AgdaSpace{}%
\AgdaOperator{\AgdaFunction{∙ₐ}}\AgdaSpace{}%
\AgdaBound{x}\AgdaSpace{}%
\AgdaOperator{\AgdaFunction{≈A}}\AgdaSpace{}%
\AgdaBound{x}\<%
\\
%
\>[8]\AgdaField{compₐ}\AgdaSpace{}%
\AgdaSymbol{:}\AgdaSpace{}%
\AgdaSymbol{∀}\AgdaSpace{}%
\AgdaBound{g'}\AgdaSpace{}%
\AgdaBound{g}\AgdaSpace{}%
\AgdaBound{x}\AgdaSpace{}%
\AgdaSymbol{→}\AgdaSpace{}%
\AgdaBound{g'}\AgdaSpace{}%
\AgdaOperator{\AgdaFunction{∙ₐ}}\AgdaSpace{}%
\AgdaBound{g}\AgdaSpace{}%
\AgdaOperator{\AgdaFunction{∙ₐ}}\AgdaSpace{}%
\AgdaBound{x}\AgdaSpace{}%
\AgdaOperator{\AgdaFunction{≈A}}\AgdaSpace{}%
\AgdaSymbol{(}\AgdaBound{g'}\AgdaSpace{}%
\AgdaOperator{\AgdaFunction{∙}}\AgdaSpace{}%
\AgdaBound{g}\AgdaSymbol{)}\AgdaSpace{}%
\AgdaOperator{\AgdaFunction{∙ₐ}}\AgdaSpace{}%
\AgdaBound{x}

\end{code}
\end{AgdaAlign}
\end{frame}
  \begin{frame}\frametitle{Acci/'on}
    \begin{block}{Acción}
        Una acción de un grupo $G$ en un conjunto $S$ es una función 
        $\cdot : G \times S \to S$, $(g, s) \mapsto g \cdot s $  tal que
        \begin{itemize}
            \item[$i)$] $(gh)\cdot s = g \cdot (h \cdot s); \forall g, h \in G$ y $\forall s \in S$.
            \item[$ii)$] $e \cdot s = s; \forall s \in S$, donde $e$ es el \textbf{elemento identidad} de $G$.
        \end{itemize}
    \end{block}
\begin{AgdaAlign}
\begin{code}%
\>[0]\AgdaKeyword{record}\AgdaSpace{}%
\AgdaRecord{IsAction}\AgdaSpace{}%
\AgdaSymbol{(}\AgdaBound{F}\AgdaSpace{}%
\AgdaSymbol{:}\AgdaSpace{}%
\AgdaRecord{Func}\AgdaSpace{}%
\AgdaSymbol{(}\AgdaFunction{G.setoid}\AgdaSpace{}%
\AgdaOperator{\AgdaFunction{×ₛ}}\AgdaSpace{}%
\AgdaBound{A}\AgdaSymbol{)}\AgdaSpace{}%
\AgdaBound{A}\AgdaSymbol{)}\AgdaSpace{}%
\AgdaSymbol{:}\AgdaSpace{}%
\AgdaPrimitive{Set}\AgdaSpace{}%
\AgdaOperator{\AgdaFunction{\AgdaUnderscore{}}}\AgdaSpace{}%
\AgdaKeyword{where}\<%
\\
\>[0][@{}l@{\AgdaIndent{0}}]%
%
\>[6]\AgdaOperator{\AgdaFunction{\AgdaUnderscore{}∙ₐ\AgdaUnderscore{}}}\AgdaSpace{}%
\AgdaSymbol{:}\AgdaSpace{}%
\AgdaField{Carrier}\AgdaSpace{}%
\AgdaBound{G}\AgdaSpace{}%
\AgdaSymbol{→}\AgdaSpace{}%
\AgdaField{Carrier}\AgdaSpace{}%
\AgdaBound{A}\AgdaSpace{}%
\AgdaSymbol{→}\AgdaSpace{}%
\AgdaField{Carrier}\AgdaSpace{}%
\AgdaBound{A}\<%
\\
%
\>[6]\AgdaOperator{\AgdaFunction{\AgdaUnderscore{}∙ₐ\AgdaUnderscore{}}}\AgdaSpace{}%
\AgdaBound{g}\AgdaSpace{}%
\AgdaBound{x}\AgdaSpace{}%
\AgdaSymbol{=}\AgdaSpace{}%
\AgdaField{Func.f}\AgdaSpace{}%
\AgdaBound{F}\AgdaSpace{}%
\AgdaSymbol{(}\AgdaBound{g}\AgdaSpace{}%
\AgdaOperator{\AgdaInductiveConstructor{,}}\AgdaSpace{}%
\AgdaBound{x}\AgdaSymbol{)}\<%
\\
%
\>[6]\AgdaKeyword{field}\<%
\\
\>[6][@{}l@{\AgdaIndent{0}}]%
\>[8]\AgdaField{idₐ}\AgdaSpace{}%
\AgdaSymbol{:}\AgdaSpace{}%
\AgdaSymbol{∀}\AgdaSpace{}%
\AgdaBound{x}\AgdaSpace{}%
\AgdaSymbol{→}\AgdaSpace{}%
\AgdaFunction{ε}\AgdaSpace{}%
\AgdaOperator{\AgdaFunction{∙ₐ}}\AgdaSpace{}%
\AgdaBound{x}\AgdaSpace{}%
\AgdaOperator{\AgdaFunction{≈A}}\AgdaSpace{}%
\AgdaBound{x}\<%
\\
%
\>[8]\AgdaField{compₐ}\AgdaSpace{}%
\AgdaSymbol{:}\AgdaSpace{}%
\AgdaSymbol{∀}\AgdaSpace{}%
\AgdaBound{g'}\AgdaSpace{}%
\AgdaBound{g}\AgdaSpace{}%
\AgdaBound{x}\AgdaSpace{}%
\AgdaSymbol{→}\AgdaSpace{}%
\AgdaBound{g'}\AgdaSpace{}%
\AgdaOperator{\AgdaFunction{∙ₐ}}\AgdaSpace{}%
\AgdaBound{g}\AgdaSpace{}%
\AgdaOperator{\AgdaFunction{∙ₐ}}\AgdaSpace{}%
\AgdaBound{x}\AgdaSpace{}%
\AgdaOperator{\AgdaFunction{≈A}}\AgdaSpace{}%
\AgdaSymbol{(}\AgdaBound{g'}\AgdaSpace{}%
\AgdaOperator{\AgdaFunction{∙}}\AgdaSpace{}%
\AgdaBound{g}\AgdaSymbol{)}\AgdaSpace{}%
\AgdaOperator{\AgdaFunction{∙ₐ}}\AgdaSpace{}%
\AgdaBound{x}

\end{code}
\end{AgdaAlign}
\end{frame}
  \begin{frame}\frametitle{GSet}
    \begin{block}{$G$-Set}
        Un conjunto $S$ en el cual un grupo $G$ tiene una acci\'on se llama $G$-set.
    \end{block}
\begin{AgdaAlign}
\begin{code}%
\>[0]\AgdaKeyword{record}\AgdaSpace{}%
\AgdaRecord{GSet}\AgdaSpace{}%
\AgdaSymbol{:}\AgdaSpace{}%
\AgdaPrimitive{Set}\AgdaSpace{}%
\AgdaOperator{\AgdaFunction{\AgdaUnderscore{}}}
\AgdaKeyword{where}\<%
\\
\>[0][@{}l@{\AgdaIndent{0}}]%
\>[2][@{}l@{\AgdaIndent{0}}]%
\>[5]\AgdaKeyword{field}\<%
\\
\>[5][@{}l@{\AgdaIndent{0}}]%
\>[7]\AgdaField{set}\AgdaSpace{}%
\AgdaSymbol{:}\AgdaSpace{}%
\AgdaRecord{Setoid}\AgdaSpace{}%
\AgdaBound{ℓ₁}\AgdaSpace{}%
\AgdaBound{ℓ₂}\<%
\\
%
\>[7]\AgdaField{action}\AgdaSpace{}%
\AgdaSymbol{:}\AgdaSpace{}%
\AgdaRecord{Func}\AgdaSpace{}%
\AgdaSymbol{(}\AgdaFunction{G.setoid}\AgdaSpace{}%
\AgdaOperator{\AgdaFunction{×ₛ}}\AgdaSpace{}%
\AgdaField{set}\AgdaSymbol{)}\AgdaSpace{}%
\AgdaField{set}\<%
\\
%
\>[7]\AgdaField{isAction}\AgdaSpace{}%
\AgdaSymbol{:}\AgdaSpace{}%
\AgdaRecord{IsAction}\AgdaSpace{}%
\AgdaField{action}

\end{code}
\end{AgdaAlign}
\end{frame}
  \begin{frame}
    \frametitle{Morfismo de $G$-Set}
    
    \begin{block}{Morfismo de $G$-Set}
        Sean $S$ y $T$ dos $G$-sets. Un morfismo de $G$-sets es función
        $f : S \to T$ tal que
        $$ f (g \cdot  x) = g \cdot f(x)$$
        para todo $x\in S$ y $g \in G$.
    \end{block}

    Estas se llaman funciones \textbf{equivariantes}.
    
    Dado que $id_{S}$ es equivariante y la composición de funciones equivariante
    resulta ser una función equivariante podemos hablar de la categoría de $G$-Sets.
    
    Cualquier conjunto $S$ puede ser visto como un $G$-set dejando $g \cdot x = x$,
    tal conjunto $G$ se llama conjunto $G$ discreto. Además, cualquier grupo actúa
    sobre sí mismo por la multiplicación.    	
\end{frame}
  \begin{frame}\frametitle{Funciones equivariantes}
\begin{AgdaAlign}
\begin{code}%
\>[0]\AgdaKeyword{record}\AgdaSpace{}%
\AgdaRecord{Equivariant}\<%
\\
\>[2][@{}l@{\AgdaIndent{0}}]%
\>[4]\AgdaSymbol{(}\AgdaBound{A}\AgdaSpace{}%
\AgdaSymbol{:}\AgdaSpace{}%
\AgdaRecord{GSet}\AgdaSpace{}%
\AgdaSymbol{)}\<%
\\
%
\>[4]\AgdaSymbol{(}\AgdaBound{B}\AgdaSpace{}%
\AgdaSymbol{:}\AgdaSpace{}%
\AgdaRecord{GSet}\AgdaSpace{}%
\AgdaSymbol{)}\AgdaSpace{}%
\AgdaSymbol{:}\AgdaSpace{}%
\AgdaPrimitive{Set}\AgdaSpace{}%
\AgdaOperator{\AgdaFunction{\AgdaUnderscore{}}}
\AgdaKeyword{where}\<%
\\
\>[0][@{}l@{\AgdaIndent{0}}]%
\>[2][@{}l@{\AgdaIndent{0}}]%
\>[5]\AgdaKeyword{field}\<%
\\
\>[4][@{}l@{\AgdaIndent{0}}]%
\>[6]\AgdaField{F}\AgdaSpace{}%
\AgdaSymbol{:}\AgdaSpace{}%
\AgdaRecord{Func}\AgdaSpace{}%
\AgdaSymbol{(}\AgdaField{set}\AgdaSpace{}%
\AgdaBound{A}\AgdaSymbol{)}\AgdaSpace{}%
\AgdaSymbol{(}\AgdaField{set}\AgdaSpace{}%
\AgdaBound{B}\AgdaSymbol{)}<%
\\
%
\>[6]\AgdaField{isEquivariant}\AgdaSpace{}%
\AgdaSymbol{:}\AgdaSpace{}%
\AgdaFunction{IsEquivariant}\AgdaSpace{}%
\AgdaSymbol{(}\AgdaField{action}\AgdaSpace{}%
\AgdaBound{A}\AgdaSymbol{)}\AgdaSpace{}%
\AgdaSymbol{(}\AgdaField{action}\AgdaSpace{}%
\AgdaBound{B}\AgdaSymbol{)}\AgdaSpace{}%
\AgdaField{F}\

\end{code}
\end{AgdaAlign}
\end{frame}
      \begin{frame}
	\frametitle{Representaciones, acciones lineales y m\'odulos}
	Sea $V$ un espacio vectorial sobre el cuerpo $\mathbb{F}$ y $G$ un grupo.
	\begin{block}{Acción Lineal}
		Una acción de un grupo $G$ en un espacio vectorial $V$ es una función 
		$\cdot : G \times V \to V$, $(g, v) \mapsto g \cdot v $  tal que
		\begin{itemize}
			\item[$i)$] $(gh)\cdot v = g \cdot (h \cdot v); \forall g, h \in G$ y $\forall v \in V$.
			\item[$ii)$] $e \cdot v = v; \forall v \in V$, donde $e$ es el \textbf{elemento identidad} de $G$.
			\item[$iii)$] $g \cdot (u+v) = g \cdot u + g \cdot v; \forall g \in G$ y $u,v \in V$.
			\item[$iv)$] $g \cdot (\lambda v) = \lambda(g \cdot v); \forall  g \in G, v \in V$ y $\forall \lambda \in F$.
		\end{itemize}
	\end{block}
	\textbf{Observación 1}. Notar que para definir una acción de $G$ en $S$ 
	los items $iii)$ y $iv)$ no son requeridos en un conjunto.
    \end{frame}

    \begin{frame}
	\frametitle{Representaciones, acciones lineales y m\'odulos}
	Luego, una representaci\'on de un grupo en un espacio vecotriales equivalente a una acci\'on del
	grupo en el espacio vectorial.
	\begin{block}{$G$-m\'odulo}
		Un espacio vectorial $V$ en el cual un grupo $G$ tiene una acci\'on lineal se llama $G$-m\'odulo.
	\end{block}
    \begin{block}{$G$-subm\'odulo}
    	Un subm\'odulo de un $G$-m\'odulo $V$ es un subespacio vectorial $U$ de $V$ tal que
    	$g \cdot u \in U$ para todo $g \in G$ y $u \in U$.
    \end{block}
    \end{frame}

    \begin{frame}
        \frametitle{Representaciones, acciones lineales y m\'odulos}
        
        \begin{block}{Representaci\'on por Permutac\'on}
            Una representaci\'on por permutac\'on de un grupo $G$ en un conjunto $S$ un
            momorfismo de $G$ en el conjunto de todas las permutaciones de $S$.
        \end{block}
        
        \begin{block}{Representaci\'on Lineal }
            Una representaci\'on lineal de un grupo $G$ en un espacio vectorial $V$ es un
            morfismo de $G$ en el grupo de todas las transformaciones lineales inversibles en $V$.
        \end{block}
        
        A menos que los califiquemos con alg\'un otro adjetivo, representaci\'on significar\'a en este trabajo
        representaci\'on lineal. Restringiremos nuestra atenci\'on en grupos finitos y espacios vectoriales sobre
        el cuerpo complejo.
    \end{frame}
    

    \begin{frame}
	\frametitle{Representaciones, acciones lineales y m\'odulos}
	\begin{theorem}%[An important theorem]
	    Dada una acci\'on de un grupo $G$ en un espacio vectorial $V$, para cada $g$ en $G$
	    definimos una funci\'on $\rho g : V \to V$ dada por $(\rho g)v = g \cdot v$ para todo $v \in V$.
	    Luego $\rho g$ es una transformaci\'on lineal inversible, y la funci\'on $\rho$ definida
	    por $g \mapsto \rho g$ es un homomorfismo de $G$ en el grupo de todas las transformaciones
	    lineales invertibles en $V$.
	    
	    Rec\'iprocamente, dado un homomorfismo $\rho$ de $G$ en el
	    grupo de todas las tranformaciones lineales inversibles en $V$, la f\'ormula
	    $g \cdot v = (\rho g)v$, define una acci\'on de $G$ en $V$.
	\end{theorem}
    \end{frame}

    \begin{frame}
	\frametitle{Representaciones, acciones lineales y m\'odulos}
    \begin{block}{Morfismo de $G$-m\'odulos}
    	Si $U$ y $V$ son $G$-m\'odulos. Un $G$-homomorfismo de $U$ en $V$ es una transformaci\'on
    	lineal $f : U \to V$ tal que $$ f(g \cdot u) = g \cdot (fu)$$  para todo $g \in G$ y $u \in U$.
    \end{block}
%    De nuevo, podemos hablar de la categoría de $G$-m\'odulo.
    \end{frame}
\end{document}
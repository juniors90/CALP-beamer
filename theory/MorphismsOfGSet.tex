\begin{frame}
    \frametitle{Morfismo de $G$-Set}
    
    \begin{block}{Morfismo de $G$-Set}
        Sean $S$ y $T$ dos $G$-sets. Un morfismo de $G$-sets es función
        $f : S \to T$ tal que
        $$ f (g \cdot  x) = g \cdot f(x)$$
        para todo $x\in S$ y $g \in G$.
    \end{block}

    Estas se llaman funciones \textbf{equivariantes}.
    
    Dado que $id_{S}$ es equivariante y la composición de funciones equivariante
    resulta ser una función equivariante podemos hablar de la categoría de $G$-Sets.
    
    Cualquier conjunto $S$ puede ser visto como un $G$-set dejando $g \cdot x = x$,
    tal conjunto $G$ se llama conjunto $G$ discreto. Además, cualquier grupo actúa
    sobre sí mismo por la multiplicación.    	
\end{frame}